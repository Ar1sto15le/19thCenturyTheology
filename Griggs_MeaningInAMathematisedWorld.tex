\documentclass[12pt]{article}
% \usepackage{inputenc}
\usepackage[top=1in, bottom=1in, left=1in, right=1in]{geometry}
\usepackage{setspace}
\doublespacing
\usepackage{verbatim}
\usepackage{parskip}
\setcounter{secnumdepth}{1}
\pagestyle{myheadings}
\usepackage{graphicx}
\usepackage{amsmath}
\usepackage{amssymb}
%\usepackage{fontspec}
%\setmainfont{Georgia}
\setlength{\parindent}{.5cm}
% These next three lines hide the citation keys in the bibliography, pretty cool.
%\makeatletter
%\def\@biblabel#1{}
%\makeatother
% These lines define the hanging indent! NEAT!
\makeatletter
% \renewcommand\@biblabel[1]{#1} % No brackets for the references
\def\@biblabel#1{}
\renewenvironment{thebibliography}[1]
     {\section*{\refname}%
      \@mkboth{\MakeUppercase\refname}{\MakeUppercase\refname}%
      \list{\@biblabel{\@arabic\c@enumiv}}%
           {\settowidth\labelwidth{\@biblabel{#1}}%
            \leftmargin\labelwidth
            \advance\leftmargin20pt% change 20 pt according to your needs
            \advance\leftmargin\labelsep
            \setlength\itemindent{-20pt}% change using the inverse of the length used before
            \@openbib@code
            \usecounter{enumiv}%
            \let\p@enumiv\@empty
            \renewcommand\theenumiv{\@arabic\c@enumiv}}%
      \sloppy
      \clubpenalty4000
      \@clubpenalty \clubpenalty
      \widowpenalty4000%
      \sfcode`\.\@m}
     {\def\@noitemerr
       {\@latex@warning{Empty `thebibliography' environment}}%
      \endlist}
\renewcommand\newblock{\hskip .11em\@plus.33em\@minus.07em}
\makeatother

\usepackage[compact]{titlesec}  
\titlespacing{\section}{0pt}{0pt}{0pt}
\titleformat*{\section}{\large\bfseries\filcenter}
\titleformat*{\subsection}{\normalsize\bfseries}
\titlespacing{\subsection}{0pt}{0pt}{0pt}
\titleformat*{\paragraph}{\footnotesize\bfseries}






\usepackage[hidelinks]{hyperref} % for urls in bibliography
\renewcommand\UrlFont{\rmfamily} % redefi

\begin{document}
% Manual Heading
\singlespacing
{\raggedleft{}Gabriel Griggs} \\
Professor Cyril O'Regan \\
19th Century Theology \\
Friday, March 4th 2016\\

\section*{Meaning in a Mathematised World: Kant and Dostoevsky on Theodicy (Through the Lens of L'Ubica Ucnik)}

\doublespacing
\subsection*{The Mathematised World: A Turn Towards Husserl}

In this paper, I will be entering into conversation with L'ubica Ucnik's paper titled ``The Problem of Morality in a Mathematised Universe: Time and Eternity in Dostoevsky's \emph{The Brothers Karamazov} and the Concept of `Love' in Patocka's Last Essay.'' Despite its absence from the title, Kant's essay on theodicy features heavily in this paper and is set in contrast to the response of theodicy given by Dostoevsky in \emph{The Brothers Karamazov}. Both of these accounts, according to Ucnik, are heavily influenced by what she terms the `mathematised world', in line with Husserl who gives a philosophical-historical analysis of the movement towards conceiving the world as a mathematically constructed and determinable object which undergirds the positive scientific endeavor. Ultimately, the question which Ucnik asks is this: can there be meaning in a mathematised world? She evaluates the answers of Kant and Dostoevsky before turning to the work of the Czech phenomenologist Patocka. In order to enter into conversation, we must have a proper background in what it means for the universe to be mathematised. In order to do so, we will turn to Husserl's essay ``The Crisis of European Sciences and Transcendental Phenomenology.'' 

Before doing so, however, I will lay out the argument of this paper. In short, the question of theodicy will be used as a test case because it sharpens the question of meaning so acutely. Kant and Dostoevsky both propose `answers' to theodicy which have to do with being in proper relation to the divine. It is my view that Kant's understanding of what it means to be in relation to the divine is insufficient in that it is an inadequate account of relationship. Furthermore, I will argue that Ucnik reduces Dostoevsky's answer too drastically to that of posing eternity as the motivation for responsible, moral behavior; instead, I will present an interpretation in which Dostoevsky's \emph{Brothers Karamazov} is taken as a guide to entering into right relationship with God. In all of this, the Book of Job will be used as the specific test case of theodicy as this will allow for fruitful dialogue between Kant, Dostoevsky, Ucnik and myself. Ultimately, the point that I would like to make is this: any notion of meaning is arbitrary, and thereby superfluous, unless it is grounded in the underlying rationality of the world itself who is the \emph{Logos}. In this particular paper, I will narrow that argument to the claim that Dostoevsky's Zosima provides a framework in which meaning can be found in a mathematised world.

\paragraph*{First Fragments of Universal Positive Science}

	From where does this language of the mathematised universe come? One place in particular is in the work of Husserl, who wonders whether the world, ``and human existence in it, truthfully have a meaning if the sciences recognize as true only what is objectively established'' in the fashion of the positive sciences which focus merely on the world as fact and treat objective truth as ``exclusively a matter of establishing what the world, the physical as well as the spiritual world, is in fact.''\footnote{Husserl, 6.} Husserl gives a historical account of this transition from a science which was not positivist in nature to one that ``rigorously grounded truth''\footnote{Husserl, 7.} in the objectivity of the positive sciences of mere correlation, where as Kant warns ``our use of the concepts of cause and effect cannot be extended beyond nature''\footnote{Ucnik, 74.} and which is really more of a product of the way we think than it is an actual phenomenon in objective reality. The first fragments of a drive for a universally valid positivistic method for investigating the world began in the mathematics of the Greeks in which an ``immense change of meaning'' took place ``whereby \emph{universal} tasks were set, primarily for mathematics \ldots tasks of a style which was \emph{new in principle}, unknown to the ancients.''\footnote{Husserl, 21.} Furthermore, with Euclidean geometry arises the important concept of an axiomatic science: ``with Euclidean geometry had grown up the highly impressive idea of a systematically coherent deductive theory, aimed at a most broadly and highly conceived ideal goal, resting on `axiomatic' fundamental concepts and principles, proceeding according to apodictic arguments -- a totality formed of pure rationality, a totality whose unconditioned truth is available to insight and which consists exclusively of unconditioned truths recognized through immediate and mediate insight.''\footnote{Husserl, 21.} 


	\paragraph*{The Conquest of the Infinite} This was only the beginning, he tells us, for the modern period brings with it ``the actual discovery and conquest of the infinite mathematical horizons'' which eventually lead to a rationalization of the natural sciences creating the ``completely new idea of \emph{mathematical natural science}.''\footnote{Husserl, 22 - 23.} This new mathematised natural science finds its first practitioner, Husserl tells us, in Galileo whose mathematical natural science led to \emph{nature itself} being ``idealized under the guidance of the new mathematics; [such that] nature itself becomes \ldots a mathematical manifold.''\footnote{Husserl, 23.} Husserl asks and answers our own question: ``\emph{What is the meaining of this mathematization of nature?}''\footnote{Husserl, 23.} He tells us that before the dominance of the positivistic, scientific mindset the world was given in ``everyday sense-experience'' in a ``subjectively relative way,'' such that we have discrepancies in the ways in which we see the world -- but, despite this, ``we believe in \emph{the} world, whose things only appear to us differently but are the same.''\footnote{Husserl, 23.} Essentially, as a result of pursuing knowledge in this shared world, that is, in pursuing knowledge of the `true' and objective world, humanity was diverted by the ``empirical art of measuring and its empirically, practically objectivizing function, through a change from the practical to theoretical interest'' as it went from investigating mathematics for practical purposes to investigating objects in the mathematical realm.\footnote{Husserl, 28.} In this turn, the empirical art of measuring ``was idealized and thus turned into the purely geometrical way of thinking.''\footnote{Husserl, 28.} It is easy to see the appeal of such an approximate science in that it yields some approximation of how things are, but only an approximation ``beginning with what is empirically given, to the geometrical ideal shape which functions as a guiding pole.''\footnote{Husserl, 28.}

	At this point in the transition, the empirical tool of measuring gives us an approximation of the real world which we know to be held together by a sort of ``\emph{universal causal regulation}'' such that ``\emph{all that is together in the world} has a universal immediate or mediate way of \emph{belonging together}\ldots''\footnote{Husserl, 28.} We have this knowledge of the world as an ``all-encompassing unity, a \emph{whole} (even though it is finite),'' in contrast to the world as being merely an aggregation of facts or ``mere totality,'' even in the life of ``prescientific knowing.''\footnote{Husserl, 31.} The transformation to `scientific knowing' is completed (and meaningful) ``only if a method can be devised of \emph{constructing}, systematically and in a sense in advance, the world, the infinitude of causalities'' from a small set of axiom-like propositions which have been established directly from experience.\footnote{Husserl, 32.} On top of this requirement is added that of being able to verify this construction ``in spite of the infinitude [of experience].''\footnote{Husserl, 32.} It is at this point that mathematics leads the way through its creation of ideal objects and principles which have been established in experience -- thereby constructing an ideal world in which we have apodictic certainty of ``true being-in-itself'' because this world has been ``apodictically generated.''\footnote{Husserl, 32.} 

	\paragraph*{From Mathematics to Natural Science} This turn towards the ideal, constructed realm does not stay contained within the realm of mathematics. Instead, due to Galileo's confidence in the possibility of achieving ``an objective science of the world,'' he pursued an idea of nature ``which is construvtively determinable in the same manner in all its \emph{other aspects.}''\footnote{Husserl, 33.} In so doing, the turn is complete and nature is conceived of as in itself being mathematically constructible. This all-encompassing mathematisation gains its power not just as a descriptive tool; it also has predictive power for ``if one has the formulae, one already possesses, in advance, the practically desired prediction of what is to be expected with empirical certainty in the intuitively given world of concretely actual life, in which mathematics is merely a special [form of] praxis.''\footnote{Husserl, 43.} Husserl then asserts that mathematisation, ``with its realized formulae, is the achievement which is decisive for life.''\footnote{Husserl, 43.} To spell this out more concretely, Husserl explains how this works:
	
	\begin{quote}
	\singlespacing
	\footnotesize
	
	In geometrical and natural-scientific mathematization, in the open infinity of possible experiences, we measure the life-wrold - the world ocnstantly given to us as actual in our concrete world-life - for a well-fitting \emph{garb of ideas}, that of the so-called objectively scientific truths. That is, through a method which (as we hope) can be realy carried out in every particular and constantly verified, we first construct numerical indices for the actual and possible sensible plena of the concretely intuited shpaes of the life-world, and in this way we obtain possibilities of predicting concrete occurrences in the intuitively given life-world, occurrences which are not yet or no longer actually given. \ldots It is through the garb of ideas that we take for \emph{true being} what is actually a \emph{method} - a method which is designed for the purpose of progressively improving, \emph{in infinitum} \ldots.\footnote{Husserl, 51.}
	
	\end{quote}
	
	 It is precisely this mathematisation which is at work in the background of Kant's project and, as Ucnick suggests, in the project of Dostoevsky with Ivan's Euclidean geometry. That this is so can be seen in the way in which Kant's project takes for granted the dualism of a natural, self-enclosed world and the psychic world.\footnote{Husserl, 60.} And, furthermore, it led to the rationalization of man's self-understanding in relationship to God for ``the philosopher, in correlation with his mathematization of the world and of philosophy, has in a certain sense mathematically idealized himself and, at the same time, God [ who has become the ] `infinitely distant man'\thinspace''.\footnote{Husserl, 66.} In a realm in which we are both able to construct and predict reality, and in fact with great success as the positive sciences have shown, the question emerges: what meaning can there be in this world? This problem becomes especially acute in suffering; for who has not been led to ask why in the midst of suffering?
	 
	 \paragraph*{A Few Notes on Husserl's Mathematisation Narrative}
	 
	 Given the insights of the narrative of mathematisation which Husserl gives us, it is tempting to absolutize and over-generalize the phenomena of mathematisation. At the very least, it should be noted that while there may be a strong tendency for natural sciences to move in the mathematical direction, it is not always the case that sciences understand themselves in terms of mathematisation in the same way that Galileo (might be taken to have) understood the reality of nature itself to be mathematical nor has it been the case that a suitable mathematics has always been ready on-hand for use by the natural sciences.\footnote{My comments on this are guided, in part, by a conversation that I had with Prof. Don Howard on mathematisation.}
	 
	 To cite a specific example, I take the case of mathematical modeling in ecology. Modeling in these cases is used to predict population growth and decline of species and to find points of instability at which these increases or decreases might happen more rapidly until settling again at another equilibrium point. It is not clear to me in reading the work of these scientists that underlying their perspective is the view that the world is, in reality, mathematical; it is equally likely that they view mathematics as a merely descriptive tool in making predictions.\footnote{The particular study that I have in mind can be found here: \url{http://limnology.wisc.edu/personnel/batt/uploads/4/6/2/3/46238055/carpenter_et_al._2011_science.pdf}. An accompanying article on this study can be found here: \url{https://www.quantamagazine.org/20151117-natures-critical-warning-system/}.} Furthermore, the science of ecology is not entirely a mathematical study in which measurements are taken so as to induce mathematical laws which are descriptive of those measurements. Instead, it is dependent on biology which is a science that is not entirely mathematical in the way that physics is. Nonetheless, Husserl's point is well taken if we broaden it to this: certainty can be found in the measurement and quantification of correlations. In this quest for certainty, there is a necessary preclusion of anything which is not quantifiable.\footnote{I myself have found an interesting model of positive science in the work of Pierre Duhem who suggests that the mathematical drive of science is not indicative of reality itself being mathematical, while at the same time maintaining that the fact of mathematisation reflects an underlying ontological ordering of reality.}\thinspace \footnote{Furthermore, I would like to thank you for our conversation with regards to further studies as it relates to this very topic. The particular question that I am interested in answering is this: how do we find the balance between benefiting from the positive sciences without completely quantifying man? My sense is that this is one of the foremost intersections of theology and science. In particular, I have in mind the question of economics: how do we as Christians learn from the positive science of economics while at the same time keeping these in balance with the teachings of the Church on the dignity of the human person?}
	 
	 

	\subsection*{Kant on Theodicy: Grounding Moral Goodness in a Stipulative God}
	
	Kant gives us his answer to the question of Theodicy in a short essay titled ``On the miscarriage of all philosophical trials in theodicy.''\footnote{Kant, Immanuel. \emph{Religion within the Boundaries of Mere Reason and Other Writings.} Trans. George Di Giovanni and Allen Wood. Ed. George Di Giovanni and Allen Wood. New York: Cambridge University Press, 2006. 17.} In attempting to come to grips with Kant's understanding of theodicy, we must first have some sense of his conception of God. Ann Loades suggests that we can gain this conception by realizing the connection to Leibniz and his optimism about God as a good being. Theodicy was \emph{the} problem by which Leibniz chose to represent his work.\footnote{``The \emph{Theodicy} itself was the one major work out of the many papers and books Leibniz produced by which he chose that most of his contemporaries should know him.'' Ann Loades, 362.} Ann Loades suggests that ``Kant's handling of the problem of evil in all its aspects is one element in his life-long preoccupation with the work of Leibniz.''\footnote{Loades, 361.} The difference between Kant and Leibniz was that Leibniz maintained that it was determinable that \emph{this} world was the best of all possible worlds, whereas Kant could not state this confidently. 
	
	Kant's skepticism appears, in part, to be a result of the careful attention he paid to the natural sciences. Loades suggests that as a result of the way in which the scientific study of nature progressed, Kant changed the way in which he thought about theology as it particularly related to theodicy.\footnote{Loades, 366.} Ultimately, he ``reached the point where he had to refuse to concede to human knowledge of creatures a replication in some sense of the divine knowledge of creatures.''\footnote{Loades, 366.} 
	
	Kant, then, is balancing two positions: on the one hand, he has the strong moral imperative to behave ethically while on the other hand he is completely skeptical of our ability to know God. The challenge, then, is to ground this morality in \emph{something} which is not quite God. He lands on the mere stipulative idea of God. Thus, despite the complete inability of man to know anything of God's ways, Kant still maintains that the stipulative idea of God is necessary for morality. This suggests a certain optimism about the goodness of God which is apparent in his insistence on a faith founded on morality. If faith was not in the same God known by our practical reason as a moral being, then there would be a drastic disconnect between the `goodness' of morality and the `goodness' of God. The one assures our confidence in the other, even if our conception of God is merely stipulative. In Kant's case, the rational imperative to behave morally assures confidence in the goodness of God because it requires a stipulative idea of God to be meaningful.

	
	In this drastic separation between the knowledge of man and the knowledge of God, one gets the sense that Kant is fighting to maintain some semblance of meaning in a mathematised world. Furthermore, we get the sense that Kant would like to maintain the faith of Leibniz in the goodness of God, despite the appearances of the world to the contrary.\footnote{Loades, 374 - 375.} In order to maintain these claims, Kant suggests that `what the study of nature and of man teaches us sufficiently elsewhere may well be true here also; that the unsearchable wisdom by which we exist is not less worthy of admiration in what it has denied than in what it has granted.'\thinspace''\footnote{Kant quoted in Loades, 369.}  Here we see a space reserved for God and man, apart from the scientific, deterministic, mathematised world. This space is necessary to maintain both the possibility of and the faith to act morally: ``this harmony required by morality `absolutely cannot as the \emph{Critique} shows, be conceived from the nature of the beings in the universe. Rather, as an agreement which for us at least is accidental, it can only be conceived through an intelligent first cause.'\thinspace''\footnote{Kant quoted in Loades, 371.} In this way, we have a preliminary answer to how meaning might be maintained in a mathematised world: the separation of the phenomenal from the noumenal in such a way that the objectivity of morality is required by the subjective demand for morality.
	
	\paragraph*{Ucnik's Reading of Kant's Theodicy}  Ucnik suggests that Kant's solution to the problem of meaning in the mathematised world is that ``we must \emph{think} of God as the ground of the highest Good, which we ought to strive for if we want to live in a world that is a decent place.''\footnote{Ucnik, 75.} This `God' can only be known as a `moral being' because we have no need to postulate him in order to explain the physical world and, likewise, we have no ability to postulate him as a necessary being. For this reason, ``the proof of the existence of such a being can be none other than a moral proof.''\footnote{Kant, 18.} We see here again the distinction between the moral realm and the physical, natural realm for according to Kant ``if we recognise natural laws only --- that is, the laws of nature that science reveals --- morality is meaningless.''\footnote{Ucnik, 75.} Thus, if we want to live in a ``decent world'' and ``to think of moral laws,'' we must think of God as only a ``stipulative idea.''\footnote{Ucnik, 75.} 
	
	Ucnik further suggests that this is the result of a shift in the ``domain of knowledge'' --- we have gone from grounding our knowledge in the Divine to grounding it in the empirically measurable.\footnote{Ucnik, 75 - 76.} Any notion of the Divine is replaced by a certain kind of scientific reasoning with a formal and idealized structure such that ``nature, with its purposes and \emph{telos}, is transformed into the geometrical manifold of modern science\ldots''\footnote{Ucnik, 76.} Thus, one way of rescuing meaning is to conceive of morality as being some aspect of divine reason as Kant does in his moral maxim which Ucnik suggests is ``a combination of divine law and human \emph{Willk\"{u}r}.''\footnote{Ucnik, 77.} This moral maxim, then, constitutes an intersection between human reason and divine reason in the form of a being able to submit oneself to the unconditional moral maxim which is a representation of the divine law. In this way, knowledge can be grounded in man's own consciousness of a universal moral maxim which merely stipulates the existence of a divine moral being.
	
	The trouble for this concept of submission to the divine law is that it is possible to allow one's own conditional good to take precedence over the unconditionally good: for ``Kant, most humans sometimes even without realising it, pervert the moral maxim, placing it in the service of self-love.''\footnote{Ucnik, 77.} The question, then, is this: how does one come to the point of willingly submitting themselves to the unconditional moral maxim? One answer is to rely on the power of self-interest in the context of an eternal framework in which a last judgment will assure just rewards for one's behavior. Such an account, however, fails to account for love and is not itself worthy of love. 
	
	Ucnik explains that Kant is aware of this problem as he asserts that ``the Christian reward cannot be understood `as if it were an offer, through which the human being would be \emph{hired}, as it were, to a good course of life; for then Christianity would\ldots not be in itself worthy of love.'\thinspace''\footnote{Kant quoted in Ucnik, 79.} At this point, Ucnik touches on what seems to be a self-referencing definition of love at work in Kant's morality. Since love, for Kant, ``cannot be directed toward a person'' but rather only towards ``\thinspace`the benefactor's generosity of will' derived from `what is universally best for the world state,' \ldots to love is to love the moral law which is universally best because it is universally valid\ldots''\footnote{Kant quoted in Ucnik, 79.} In other words, love is a deterministic function whose value is given by the unconditional moral law. It is not an act of will so much as an obedience to the `function' of the unconditional law. 
	
	With this understanding of love, we get a better sense of what it would mean for Christianity to be worthy of love: Christianity is only worth of love if it intersects sufficiently with the unconditional moral maxim, or perhaps more accurately, only if it contains the whole unconditional moral maxim. And so the ground for moral behavior in the context of Christianity goes back to the intersection of the divine law with the human submission to this law which can only be known through moral reasoning. This gives us a better sense of the problem that Kant is trying to answer, namely, how morality can exist within a mathematised world, but it does not give us a sense of the motivating force behind submission to the unconditional law, as envisioned by Kant. Here it should be noted that Ucnik's use of Kant is not intended to investigate this question; rather, Kant's theodicy is a means for framing what Ucnik takes as Dostoevsky's dependence on eternity for motivating moral behavior. 
	
	\paragraph*{The Motivation for Morality} Conceptually, in the Kantian framework, we have a structure in which it would be possible for morality to co-exist with a mechanized, deterministic world. In this framework there is a separate sphere of human action which is guided by the will, the sphere of the noumenal. The question remains, however, as to what force would motivate a person to behave morally --- especially in light of the fact that so much suffering is brought directly from the malicious behavior. The answer cannot be motivated by love, at least as typically understood, because love is merely a determined product of the unconditional moral maxim, as we have seen. This suggests that the answer, in the Kantian framework, lies in some aspect of reason. In particular, it is the same kind of reason by which we come to know God as a moral being: efficacious practical reason.
	
	We can reframe this question in an important context by taking as given human free will in the noumenal sphere alongside some sort of corruption of that will, which will be articulated hereafter. In this way, the question that we are trying to answer is this: how is it that human beings are capable of doing good in light of the ``\emph{radical} innate \emph{evil} in human nature''?\footnote{Kant Religion, 56.} To answer this question, we turn to Kant's \emph{Religion within the Boundaries of Mere Reason} where we learn that ``the ground of evil cannot lie in any object \emph{determining} the power of choice through inclination, not in any natural impulses, but only in a rule that the power of choice itself produces for the exercise of its freedom, i.e., in a maxim.''\footnote{Kant Religion, 47.}

	This radical innate evil in human nature is, in some sense, just as inscrutable to us as our propensity for good: ``One cannot, however, go on asking what, in a human being, might be the subjective ground of the adoption of this maxim [good maxim] rather than its opposite [bad maxim].''\footnote{Kant Religion, 47.} Kant is almost forced to this position in order to maintain freedom by the fact that ``if this ground were ultimately no longer itself a maxim, but merely a natural impulse, the entire exercise of freedom could be traced back to a determination through natural causes\ldots.''\footnote{Kant Religion, 47.} In spite of this inscrutability, Kant persists in trying to know what he can about this propensity for evil and how it might be overcome. Ultimately, his answer is shrouded with some sense of mystery. There is on the one hand the mystery of the moral imperative for ``in spite of that fall, the command that we \emph{ought} to become better human beings still resounds unabated in our souls,'' and, on the other hand, there is a faith that ``we must also be capable of it [becoming better human beings], even if what we can do is of itself insufficient and, by virtue of it, we only make ourselves receptive to a higher assistance inscrutable to us.''\footnote{Kant, 66.} 
	
	\paragraph*{A Connection to the Divine} This confidence in being able to act morally comes from our only connection to the divine which comes through efficacious practical reason. The intersection between the divine and the human that Ucnik discussed takes place in what Kant calls ``authentic interpretation'' which we see at work in the question of theodicy. As in the question of the first subjective ground of determining our moral maxim, our ability to determine anything of God's final purposes from the world is severely limited. Despite the fact that the world, as a work of God, can ``be considered by us as a divine publication of his will's \emph{purposes},'' it is a ``closed book'' for us ``\emph{every time} we look at it to extract from it God's \emph{final aim} (which is always moral)\ldots.''\footnote{Kant, 24.} For this reason, we have no hope of inferring from the world some sort of interpretation which might construe this world as the best of all possible worlds. Instead, we must be contented by the fact that we cannot know God's final purposes while at the same time grounding our morality in the morality of the divine, which is the one aspect of the divine which we can know. 
	
	This morality of the divine is tied intimately to what Kant calls ``authentic theodicy'' which is a product of the very same sort of reason by which we legislate our moral behavior and known as ``efficacious practical reason.'' Authentic theodicy is the ``mere dismissal of all objections against divine wisdom'' which is a ``pronouncement of the same reason through which we form our concept of God --- necessarily and prior to all experience --- as a moral and wise being.''\footnote{Kant, 24.} This is the very same reason by which we legislate our moral behavior ``absolutely without further grounds.''\footnote{Kant, 24.} Importantly, in efficacious practical reason, God ``becomes himself the interpreter of his will as announced through creation'' so that morality now acts as a transcendental bridge to the divine and as a means out of the mathematised world in which Kant has been working..\footnote{Kant, 24.}  
	
	\paragraph*{Authenticity as Relationship to God} Makkreel is helpful on this point in explaining the way in which ``authenticity'' takes on a new meaning for Kant. In particular, authenticity is transformed from its original sense to a new ontological sense for Kant. In philological criticism authenticity ``literally means \emph{being} an original source'' whereas for Kant ``authenticity involves something more general, namely, \emph{having} an appropriate relation to an original source.''\footnote{Makkreel, 69.} This is borne out from what we have seen as an ``authentic interpretation'' in that ``authentic'' here indicates a relation between God and the interpreter such that God's will is being made known through the use of efficacious practical reason. This implies that in order to make such an authentic interpretation, one must be in right relationship with God such that an ``authentic theodicy becomes possible for someone who stands in proper respectful relation to God\ldots''\footnote{Makkreel, 69.} It is at this point that Makkreel examines Kant on Job, but for now, I would like to touch on a different aspect of his argument.
	
	As part of Makkreel's analysis, he argues that because ``an authentic theodicy judges the whole on the basis of a subjective or aesthetic-moral feeling, it makes sense to relate Kant's views on theodicy back to the work that precedes it: the \emph{Critique of Judgment.}''\footnote{Makkreel, 70.} In so doing, Makkreel hopes to shed light on an aspect of authentic interpretation that separates it from those judgments made by an axiomatic, scientific method. He relates two sets of pairs: doctrinal interpretation to determinant judgment and authentic interpretation to reflective judgment. These sets of pairs are distinguished from each other ``by the way in which they relate particulars to universals.''\footnote{Makkreel, 70.} In particular, Makkreel notes that the claims of science are determinant in ``that they start with universal concepts or rules and explain particulars on this basis,'' whereas reflective judgments operate in a more inductive manner, working from particulars ``to find an appropriate universal.''\footnote{Makkreel, 70.} This is an interesting intersection point with the mathematised world, but it fails to add anything to our analysis because science is not limited to axiomatic, deductive reasoning. In fact, a great deal of mathematisation happens precisely in an inductive manner. 
	
	\paragraph*{Makkreel on Job} That point aside, the concept of authenticity as a proper relationship to God sheds light on why Kant interprets Job as he does. Kant uses the story of Job as one of authentic theodicy because in this story he finds an ``authentic interpretation expressed allegorically.''\footnote{Kant, 25.} Makkreel keys in on this particular sentence in Kant's retelling of Job: ``Hence only sincerity of heart and not distinction of insight; honesty in openly admitting one's doubts; repugnance to pretending conviction where one feels none, especially before God \ldots these are the attributes which, in the person of Job, have decided the preeminence of the honest man over the religious flatterer in the divine verdict.''\footnote{Kant, 26.} If Job's relationship with God is the key to his being vindicated, we have to say that the key to his relationship with God is sincerity. In particular, Kant reads Job's relationship to God as being centered on Job's recognition of the impotence of his reason and authenticity in representing his thoughts and feelings.\footnote{Kant, 27.}
	
	That being the case, Job's vindication will not result in a new piece of knowledge; instead, it results in merely an ``acknowledgment of what he holds to be true''\footnote{Makkreel, 70.} as seen in Kant's comment that the conclusion of the affair is that God has reaffirmed the way in which Job talks about God. Keeping relationality as the interpretative key, we can say that Job is re-affirmed in his saying essentially that he can know nothing of God's ways and, that, therefore he is simply re-affirming himself as being infinitely distant from God's wisdom. 
	
	\paragraph*{A Critique of Kant's Theodicy} Given this conviction of ignorance, Kant continues to argue that Job's faith ``could only arise in the soul of a man who, in the midst of his strongest doubts, could yet say (Job 27:5-6): `Till I die I will not remove mine integrity from me, etc.'\thinspace''\footnote{Kant, 26.} In other words, authentic faith as being in right relationship with God is now a function of one's integrity which is also the character trait by which ``one can and must stand by the \emph{truthfulness} of one's declaration or confession\ldots''\footnote{Kant, 27.} Ultimately, Kant makes this turn because he would like to show that Job's faith was founded on his morality.\footnote{Kant, 26.} This suggests a picture in which Job's motivation for acting morally is merely a result of his sincerity in following the dictates of his efficacious practical reason. As we have seen, however, even Kant himself has a difficult time in formulating \emph{how exactly} man's propensity for evil is overcome so that he can be faithful to his conscience as it is known through his efficacious practical reason. Furthermore, even Kant admits some recourse to divine grace in doing so when he says that 
	
	My critique of Kant's interpretation of Job, then, is this: in founding faith on morality, Kant has failed to give an account of the motivation for maintaining one's integrity and acting morally in the first place. Furthermore, there is an \emph{implicit} faith in both the universal subjectivity of morality as well as in the inherent value of morality. In some sense, it seems as if the sort of morality that Kant is suggesting \emph{is a sort of faith} which takes for granted the ability of practical reason to be efficacious. This is suggested by Kant's inability to formulate an answer to the question of how human beings can overcome their propensity for evil. It is further suggested by what I take to be a misinterpretation of Job's encounter with God.
	
	\paragraph*{Job's Encounter with God} As Makkreel has pointed out, if Kant is to be consistent in maintaining the impotence of man's reason as the reason for Job's vindication, then it must also be the case that Job did not gain any new knowledge of God through his encounter. We hear that all ``that Job gains through his authentic response to his suffering is an acknowledgment of what he holds to be true [namely, the impotence of his reason to know God].''\footnote{Makkreel, 70.} Is this really the case? What are we to make of Job's statement ``I had heard of you, but now I see you?''\footnote{Cite this...} Was Job merely re-convicted of his position of ignorance --- perhaps he saw his ignorance in a new light? On this account, I find Kant's interpretation to be lacking and therefore I find his account of authentic theodicy ultimately to be insufficient as well. 
	
	In essence, I find it to be tantamount to saying this: ``Morality is important, but if it is not grounded in anything then it is meaningless; therefore, let us stipulate a conception of God as a moral being.'' In a very real sense, to think in this way is almost to think in a mathematically constructive manner. It is as if we have said that we would like to maintain the proposition that morality is important, so we have constructed a concept to be the foundation for this claim. So long as our constructed concept of God can be maintained in a logically consistent manner, then there is no problem. God as a merely stipulative idea, however, has no power to motivate or act in one's life in such a way that our inherent propensity towards immoral behavior can be overcome. In order to find meaning in this mathematised universe, I would like to turn, as Ucnik has done, to Dostoevsky's \emph{The Brothers Karamazov} because this novel, in my view, proposes a solution to theodicy and the grounding of reality which is actually sufficient and not merely arbitrary. Conveniently, too, this is done in the context of the Book of Job. 
	
	\subsection*{A Turn to Dostoevsky: Is Meaning in This Life a Product of Eternal Life?}
	
	As I have mentioned, Ucnik has used Kant in order to lay the foundation for her comments on Dostoevsky, whom she sees as both trying to answer the same question as Kant and as responding to Kant. She tells us that the ``Kantian puzzle, `why should people behave morally?'\thinspace'' is represented and expressed by Ivan Karamazov in \emph{The Brothers Karamazov} by his acceptance of the ``\thinspace`problematic of reason and skepticism': `There is no virtue if there is no immortality' (70).''\footnote{BK quoted in Ucnik, 81.} In fact, Ucnik even sees both the Kantian acknowledgment of God as a moral being and the Kantian position of skepticism towards our ability to know anything of God. She sees this particularly in a passage from \emph{The Brothers Karamazov} in which Ivan is explaining how it is possible for him to accept the notion of God while at the same time rejecting his world.\footnote{Ucnik, 83.} Notice, in particular, the emphasis on Euclidean and non-Euclidean geometry as well as its importance for (a priori?) conceptions within the mind. 
	
	\begin{quote}
	\singlespacing
	\footnotesize
	My task is to explain to you [Alyosha] as quickly as possible my essence, that is, what sort of man I am, what I believe in, and what I hope for, is that right? And therefore I declare that I accept God pure and simple. But this, however, needs to be noted: if God exists and if he indeed created the earth, then, as we know perfectly well, he created it in accordance with Euclidean geometry, and he created human reason with a conception of only three dimensions of space. At the same time there were and are even now geometers and philosophers, even some of the most outstanding among them, who doubt that the whole universe, or, even more broadly, the whole of being, was created purely in accordance with Euclidean geometry; they even dare to dream that two parallel lines, which according to Euclid cannot possibly meet on earth, may perhaps meet somewhere in infinity. I, my dear, have come to the conclusion that if I cannot understand even that, then it is not for me to understand about God. I humbly confess that I do not have any ability to resolve such questions, I have a Euclidean mind, an earthly mind, and therefore it is not for us to resolve things that are not of this world.\footnote{BK, 235.}
	\end{quote}
	
	At work here is a powerful example of mathematisation. It is seen in Ivan's claiming, in Kantian manner, that he has a Euclidean mind which is incapable of understanding the non-Euclidean geometries which were then coming into existence. The postulate that parallel lines would never meet is extremely important for the development of non-Euclidean geometries in that these geometries take the first four Euclidean axioms and disregard the parallel postulate. In so doing, self-consistent geometries are created which yield the interesting result that parallel lines meet in these geometries. Therefore, Ivan's claim of not understanding these particular kinds of geometries can be taken, in the language of mathematisation, as him saying that he cannot understand a particular \emph{kind} of mathematisation of the universe. In some sense, Ivan is caught in a paradigm in which he cannot see past his mathematisation. This is seen most clearly in a rigged question that he asks Alyosha:
	
	\begin{quote}
	\footnotesize
	\singlespacing
	
	Tell me straight out, I call on you---answer me: imagine that you yourself are building the edifice of human destiny with the object of making people happy in the finale, of given them peace and rest at last, but for that you must inevitably and unavoidably torture just one tiny creature, that same child who was beating her chest with her little fist, and raise your edifice on the foundation of her unrequited tears---would you agree to be the architect on such conditions?\footnote{BK, 245.}

	\end{quote}
	
	\paragraph*{Ucnik's Reading of Dostoevsky's Answer} Ucnik comments on this passage that because of Ivan Euclidean mind, we might say because of his reliance on Euclidean mathematisation, the ``answer is already implicit in the question'' and ``no other option is available.''\footnote{Ucnik, 83.} She thus proposes that Father Zosima offers a way out of this mathematised dilemma so as to also find meaning in the midst of suffering. In so doing, she notes that the switch happens as a result of ``a \emph{personal} encounter with suffering'' in which love ``ceases to be a Kantian love of God [a love of a moral maxim]'' and ``is reconfigured as a personal understanding of suffering, the love of a real suffering person and their struggles.''\footnote{Ucnik, 84.} This personalization of suffering takes on all the more meaning in the context of suffering children, which she highlights as a major framing device of the novel. At this point, we also hear of the importance of Jesus' suffering as she tells us that for Dostoevsky, ``faith in Jesus and his suffering leads to love of particular others.''\footnote{Ucnik, 84.} Interestingly, despite the importance of Jesus for Dostoevsky's position, Ucnik mentions him only one more time in the context of Alyosha who ``wholeheartedly believes in Jesus, God's love, future life, and love for others.''\footnote{Ucnik, 85.}
	
	I mention this because I would suggest that any answer to the question of finding meaning in the mathematised world in which Jesus does not feature prominently is one that will be insufficient. Ucnik is certainly not wrong in reading the answer proposed through Zosima as that of a love which is ``again a love for concrete people and things, for the world created by God.'' It is inconsistent, however, to claim that Dostoevsky's ``faith in Jesus and his suffering leads to love of particular others'' and then claim that, for Dostoevsky, ``[e]ternity is the answer to the world of suffering.''\footnote{Ucnik, 84 - 85.} She goes on to read Dostoevsky as saying that our ``belief in the immortality of the soul and in future harmony can give us strength to live here, in this world, where we must love others, not in an abstract, detached manner, but concretely.''\footnote{Ucnik, 85.} 
	
	She further suggests that Dostoevsky's exit plan from Ivan's Euclidean dilemma is to ``let go and realise that there is more to life than nature's mechanism: that there is the order of things that God guarantees.''\footnote{Ucnik, 85.} While I do not think that this claim is itself mistaken, I think her emphasis on how this \emph{letting go takes place} is mistaken. This letting go seems to be, in her mind, a function of eternal life. Somehow, the mere emphasis of eternal life is what gives us ``strength to live here'' and to see that there is more to life than Euclidean logic. She then continues to read Dostoevsky's answer as being a call to personalisation: ``Zosima, however (with Dostoevsky), sees through this problem by `personalising' his encounters with the concrete, living people that he talks to. \ldots It is these concrete relationships that might enable us to forgive and to love, and help others to do the same.''\footnote{Ucnik, 86.} Can this claim be consistent with the fact that the terrible crimes which Ivan protests are also concrete encounters, in the flesh, with living people?
	
	\paragraph*{Ucnik's Motivational Problem} My sense is that Ucnik focuses on the concrete encounters as emphasized in \emph{The Brothers Karamazov} because it ties in nicely with her answer (with Patocka) to the problem of mathematisation which she suggests is a matter of realizing that we should not follow Kant and Dostoevsky in assuming that ``there are only two options: the physical world of science, and God.''\footnote{Ucnik, 86.} In order to make this claim, she further suggests that we can come to know a third option through Socratic self-knowledge. We must come to know that ``we can never know all,'' but nonetheless we must ``strive to be true to ourselves.''\footnote{Ucnik, 87.} Presumably, if we our true to ourselves we will find within ourselves a call to responsibility: ``we must take responsibility for our actions, realising that we live in a world that is always in decline but can rise above it and assume responsibility not only for ourselves, our acts, but also for the world as we have inherited it.''\footnote{Ucnik, 87.} While I agree with this call for responsibility and self-knowledge, I echo Augustine's remarks which highlight a true motivational problem in human nature. How can we call for Socratic self-knowledge without articulating the means by which that self-knowledge is possible?
	
	\begin{quote}
	\singlespacing
	\footnotesize
	
	``Now, if some great and divine man should arise to persuade the peoples that such things were to be at least believed if they could not grasp them with the mind, or that those who could grasp them should should not allow themselves to be implicated in the depraved opinions of the multitude or to be overborne by vulgar errors, would you not judge that such a man is worthy of divine honours?''
	
	I believe that Plato's answer would be: ``That could not be done by man, unless the very virtue and wisdom of God delivered him from natural environment, illumined him from his cradle not by human teaching but by personal illumination, honoured him with such grace, strengthened hiim with such firmness and exalted him with such majesty, that he should be able to despise all that wicked men desire, to suffer all that they dread, to do all that they marvel at, and so with the greatest love and authority to convert the human race to so sound a faith. \ldots Being the bearer and instrument of the wisdom of God on behalf of the true salvation of the human race, such a man would have earned a place all his own, a place above all humanity.''\footnote{Augustine, 227.}
	
	\end{quote}
	
	Ucnik persists, however, in attempting to find such motivation for Socratic humility and self-knowledge in the world which ``is a horizon that gives meaning to our lives here and now.''\footnote{Ucnik, 87.} It does not follow immediately that the very fact of having a horizon, a time limit on our lives, implies any movement towards self-knowledge; in fact, many have and do see this time limit as a reason for living lives of debauchery. And so to justify this claim, Ucnik proposes that we are able to detect some inherent meaning in the world by way of concretized relationships: ``We are born into a family,'' she tells us, such that others ``take care of our needs and teach us how to be a person in the world.''\footnote{Ucnik, 87.} This concretization of personal relationship is an effort to realize that there is ``more to life'' than the mathematical formalism with which the natural sciences proceed: the ``world of science is not the world of our living but proceeds from it \ldots Formalised, scientific nature is not the world of our living, although it proceeds from an abstraction of the life-world, where the course of our lives runs.''\footnote{Ucnik, 87.}
	
	As we are `anchored' into these concrete, personal relationships, we begin to extend ourselves in society, to ``grow up and become a part of society.''\footnote{Ucnik, 87.} In so doing, we start to become ``ruled by the public anonym, or as Heidegger calls it, \emph{Das Man}, everybody and nobody.''\footnote{Ucnik, 87.} We find ourselves wanting to step outside the prescribed rules, but unable to as a result of not knowing how to step outside of the rules.\footnote{Ucnik, 87.} In order to step outside of these rules, Ucnik suggests that ``a tentative answer to this problem is sketched by Dostoevsky in his story, `The Dream of a Ridiculous man.'\thinspace''\footnote{Ucnik, 88.} In this story, the essential action is that of a concrete personal relationship with a little girl who prompts a dream in which he finds himself in a ``land of strange people who do not know science as we have it, who love without need to know why.''\footnote{Ucnik, 88.} This leads to what Ucnik (in the thought of Patocka) frames as the ``third movement of existence,'' that of the breakthrough in which one accepts responsibility of all and for all. And here is the crucial motivational issue at the heart of Ucnik's claim: 
	
	\begin{quote}
	\singlespacing
	\footnotesize
	
	This is not a simple task and not all of us will reach this stage in our lives. Yet, it is something we all can achieve. It is the task that we must strive for all our life. \emph{Not because of God, but because we know that we will never know all, and that despite this not-knowing, we are responsible for all.}\footnote{Ucnik, 89. [emphasis mine]}
	
	\end{quote}

	Just because we say that we can, does not mean that we are able nor does it imply that we are willing.
	
	\subsection*{Jesus, Zosima and Job: A Revelatory Answer to the Question of Meaning in a Mathematised World}
	
	As I mentioned earlier, I believe that both Kant and Ucnik have a fundamental problem: they are unable to give an account of how it is that human beings are able to overcome their fallen nature. 
	
	Kant has suggested a tentative reliance on divine grace, but in practice has now allowed it. This becomes especially clear in what I take to be an over-reliance on efficacious practical reason which is capable of forming a concept of God --- ``necessarily and priori to all experience --- as a moral and wise being.''\footnote{Kant, 24.} This efficacious practical reason is also the very means by which God ``becomes himself the interpreter of his will as announced through creation'' in such a way that the moral legislation of practical reason ``can be considered as the unmediated definition and voice of God through which he gives meaning to the letter of his creation.''\footnote{Kant, 24-25.} To these claims, I simply pose the question: if it is through efficacious practical reason that God himself imbues meaning to our lives in the realm of morality, then how is it possible that this dictate of reason can be denied in practice? Even on Kant's account of Job, it is necessary to first put oneself in right relationship with God \emph{in order to have an authentic theodicy in the first place.} This suggests that something more than efficacious practical reason is at work.
	
	On Ucnik's account, I do not know her position on the fallen nature of humanity. It seems likely, however, that such an account is absent from her ontology. We are able to become responsible for all, ``not because of God'' she tells us, but rather because we are capable of coming to ``know that we will never know all, and that despite this not-knowing, we [know that] we are responsible for all.''\footnote{Ucnik, 89.} On her account, the problem of finding meaning in the mathematised world has become a problem of ignorance --- we simply lack the proper humility to accept that we will never know all things. Augustine's remark to the Platonists of his day is particularly poignant as a rebuttal to Ucnik's position in that it highlights a fundamental claim of Christianity: overcoming our fallen nature and becoming capable of Christ-like love is not simply a matter of knowledge; instead, it is inextricably linked with the phenomenon of Christ's Incarnation, Passion and Resurrection. Despite her acknowledgment in the importance of Christ for Dostoevsky, she fails to take this into account of her reading of Dostoevsky's Zosima.
	
	\paragraph*{Crucial Aspects of a Solution} In order to give a proper account of the answer to the question of theodicy --- and ultimately the question of meaning in a seemingly deterministic, mathematised world --- there are at least three valuable insights that we can garner from the Book of Job and \emph{The Brothers Karamazov.} The first is that there is indeed evil at work in the world as seen by the Devil who asks  The second is that Job has a direct, personal encounter with God in which some new knowledge is revealed to him. 
	
	
	

\pagebreak
\begingroup
\renewcommand{\section}[2]{}	% in article, this becomes reference, so we suppress the normal \section\refname
\centerline{\textbf{Bibliography}} 
\begin{thebibliography}{9}

\bibitem{Kant}
Kant, Immanuel. \emph{Religion within the Boundaries of Mere Reason and Other Writings.} Trans. George Di Giovanni and Allen Wood. Ed. George Di Giovanni and Allen Wood. New York: Cambridge University Press, 2006. 
\end{thebibliography}

\endgroup

\end{document}
